%%%%%%%%%%%%%%%%%%%%%%%%%%%%%%%%%%%%%%%%%
% Plasmati Graduate CV
% LaTeX Template
% Version 1.0 (24/3/13)
%
% This template has been downloaded from:
% http://www.LaTeXTemplates.com
%
% Original author:
% Alessandro Plasmati (alessandro.plasmati@gmail.com)
%
% License:
% CC BY-NC-SA 3.0 (http://creativecommons.org/licenses/by-nc-sa/3.0/)
%
% Important note:
% This template needs to be compiled with XeLaTeX.
% The main document font is called Fontin and can be downloaded for free
% from here: http://www.exljbris.com/fontin.html
%
%%%%%%%%%%%%%%%%%%%%%%%%%%%%%%%%%%%%%%%%%

%----------------------------------------------------------------------------------------
%	PACKAGES AND OTHER DOCUMENT CONFIGURATIONS
%----------------------------------------------------------------------------------------

\documentclass[10pt]{article} % Default font size and paper size
\usepackage{geometry}
\geometry{top=2cm,bottom=2cm}
\usepackage{hyperref}


\usepackage{fontspec} % For loading fonts
\defaultfontfeatures{Mapping=tex-text}
%\setmainfont[SmallCapsFont = Fontin SmallCaps]{Fontin} % Main document font


\usepackage{xunicode,xltxtra,url,parskip} % Formatting packages

\usepackage[usenames,dvipsnames]{xcolor} % Required for specifying custom colors

%\usepackage{fullpage}
%\usepackage[big]{layaureo} % Margin formatting of the A4 page, an alternative to layaureo can be \usepackage{fullpage}
% To reduce the height of the top margin uncomment: \addtolength{\voffset}{-1.3cm}
\addtolength{\voffset}{-0.8cm}

\usepackage{hyperref} % Required for adding links	and customizing them
\definecolor{linkcolour}{rgb}{0,0.2,0.6} % Link color
\hypersetup{colorlinks,breaklinks,urlcolor=linkcolour,linkcolor=linkcolour} % Set link colors throughout the document

\usepackage{titlesec} % Used to customize the \section command
\titleformat{\section}{\Large\scshape\raggedright}{}{0em}{}[\titlerule] % Text formatting of sections
\titlespacing{\section}{0pt}{3pt}{3pt} % Spacing around sections

\begin{document}

\pagestyle{empty} % Removes page numbering

%\font\fb=''[cmr10]'' % Change the font of the \LaTeX command under the skills section

%----------------------------------------------------------------------------------------
%	NAME AND CONTACT INFORMATION
%----------------------------------------------------------------------------------------


\par{\centering{\Huge Etienne \textsc{Bonnassieux}}\bigskip\par} % Your name
%\vspace{1 cm}
%%\section{Personal Information}
%\hspace{-0.5cm}
%\begin{tabular}{rl}
%\textsc{Born:} & Noisy-le-Sec  | 19/10/1991 \\
%\textsc{Address:} & 15 Rue Lakanal, 75015 Paris \\
%\end{tabular}
%\hspace{2cm}
\begin{center}
\begin{tabular}{rl|rl}
\textsc{Date of birth:} & 19/10/1991 &
\textsc{email:} & \href{mailto:etienne.bonnassieux@obspm.fr}{etienne.bonnassieux@uni-wuerzburg.de}\\
\textsc{Place of Birth:} &Noisy-le-Sec &
\textsc{Phone:} & +49 17 83 46 40 75
\end{tabular}
\end{center}

\vspace{0.5cm}

%----------------------------------------------------------------------------------------
%	COLLABS
%----------------------------------------------------------------------------------------

\section{Active international collaborations}

\begin{tabular}{r|p{12.5cm}}
	
	\textsc{Feb 2022} & \textbf{DFG Research Unit: ``Relativistic Jets in Active Galaxies"}\\
	\textsc{Present}  & Specifically worked on the project to study ``Large-Scale Blazar Jets: Clues on High-Energy Emission from Low-Frequency Radio Observations".\\
	\multicolumn{2}{c}{} \\
	
	\textsc{Oct 2018} & \textbf{LOFAR-IT}\\
	\textsc{Present}  & Collaboration working to meet the requirements of the Italian LOFAR community.\\
	\multicolumn{2}{c}{} \\
	
	\textsc{Oct 2018} & \textbf{DRANOEL Working Group}\\
	\textsc{Present}  & Collaboration focusing on the study of galaxy clusters \& radio relics.\\
	\multicolumn{2}{c}{} \\

	\textsc{Oct 2017} & \textbf{NenuFAR}\\
	\textsc{Present}  & French low-$\nu$ extension of LOFAR; I head its ``Cluster Filaments \& Cosmic Magnetism" early key science project ES09.\\
	\multicolumn{2}{c}{} \\
	
	\textsc{Oct 2017} & \textbf{LOFAR-VLBI}\\
	\textsc{Present}  &  Working group tasked with developing the capabilities of international LOFAR.\\
	\multicolumn{2}{c}{} \\
	
	\textsc{Oct 2015} & \textbf{LOFAR Surveys KSP}\\
	\textsc{Present}  &  Working group tasked with creating large-scale surveys of the LOFAR radio sky.\\
	\multicolumn{2}{c}{} \\
	
	%------------------------------------------------
\end{tabular}

%----------------------------------------------------------------------------------------
%	EDUCATION
%----------------------------------------------------------------------------------------

\section{Education \& Qualifications}

\begin{tabular}{r|p{12.5cm}}
	\textsc{Jan 2020} & \textbf{Obtained CNU Qualification}\\
	& Obtained CNU Qualification under Section 34, which makes me eligible to hold lecturer positions in French universities.\\
	\multicolumn{2}{c}{} \\

\textsc{2015-2018} & \textbf{PhD in Astrophysics} - \textit{Observatoire de Paris \& Rhodes University}\\
& Supervisors: Philippe Zarka, Oleg Smirnov, Cyril Tasse\\
& ``Statistical Analysis of the Radio-Interferometric Measurement Equation,
a derived adaptive weighting scheme, and applications to LOFAR-VLBI
observation of the Extended Groth Strip"\\
& Partnership: LESIA at the Observatoire de Paris (ED127) \& RATT-RU, SKA-SA\\
\multicolumn{2}{c}{} \\

\textsc{2013-2015} & \textbf{M1 \& M2R Astronomie, Astrophysique et Ingénierie Spatiale} \\
& Equivalent to Msc. I graduated with the Astronomy \& Astrophysics program.\\
& Partnership : Observatoire de Paris, UPMC, Diderot, Orsay, ENS Ulm\\
%& Université Paris Diderot (Paris 7), Université d'Orsay (Paris 11), \\
%& École Normale Supérieure (ENS Ulm) \\
\multicolumn{2}{c}{} \\

%\textsc{2013-2014} & \textbf{M1 Astronomie et Astrophysique, "Sciences de l’Univers et Technologies Spatiales"} \\& - \textit{Observatoire de Paris} \\
%& First year of French 2-year Masters. Graduated "Mention Bien" (honours). \\
%\multicolumn{2}{c}{} \\

\textsc{2009-2013} & \textbf{Bsc (2:2, Hons) in Astrophysics} - \textit{University of Edinburgh} \\
& Bachelors of Science, graduated with Honours.\\
\multicolumn{2}{c}{} \\

\textsc{2008-2009} & \textbf{IB Diploma} - \textit{Bahrain School} \\
%& Highers: Mathematics, French, Physics, Economics. \\
%& Standard: English, Chemistry \\
\multicolumn{2}{c}{} \\

%------------------------------------------------
\end{tabular}


\section{Research Positions}

\begin{tabular}{r|p{12.5cm}}
	\textsc{Feb 2022} & Post-doctoral Fellowship studying relativistic blazar jets at low frequencies at the\\
	\textsc{Present}& Julius-Maximilians-Universit{\"a}t in W{\"u}rzburg, Germany, under the supervision of Matthias Kadler as part of a joint DFG grant with the University of Hamburg.\\
	\multicolumn{2}{c}{} \\
	\textsc{Oct 2018} & Post-doctoral Fellowship studying galactic cluster science at low frequencies at the\\
	\textsc{Feb 2022}& University of Bologna, Italy, under the supervision of Annalisa Bonafede as part of the
	DRANOEL ERC grant.\\
	\multicolumn{2}{c}{} \\
\end{tabular}

\section{Teaching \& Scientific Outreach}

\begin{tabular}{r|p{12.5cm}}
	\textsc{Jun 2019} & \textbf{Lectured at the First Italian LOFAR School}\vspace{1mm}\\
	& Taught a workshop on using modern direction-dependent calibration \& imaging
	suites DDF and killMS to participants. Helped tutor in the courses of colleagues.\\
	\multicolumn{2}{c}{} \\
	
	& \textbf{Tutored in the Paris Observatory DU-LU course}	\vspace{1mm}\\
	\textsc{Sep 2017} & Supervised four students as part of an online course, usually teachers or amateur\\
	\textsc{Jul 2018} & scientists in the workforce.\vspace{1mm}\\
	\textsc{Sep 2015} &Of my six students, four successfully carried on to other programs in the DU;\\
	\textsc{Jul 2016} & two dropped during the year for personal reasons.\\
	\multicolumn{2}{c}{} \\

	& \textbf{Lectured for NASSP}\vspace{1mm}\\
	\textsc{Sep 2017} & Interferometry course: two 1-hour lectures on visibilities, UV-plane, PSF, and
	ZVC theorem. Course was aimed at honours astrophysics students at UCT.\vspace{1mm}\\
	\textsc{Sep 2016} & As above, but aimed at masters astrophysics students at UCT: content was at a
	higher level. This also entailed writing and marking a homework question.\\
	\multicolumn{2}{c}{} \\
	

	\textsc{Sep 2017} & \textbf{Wrote and organised a pyrap tutorial during \hyperlink{http://www.ast.uct.ac.za/3gc4hifidelity/}{3GC4}}\\
	& Wrote \hyperlink{https://github.com/ebonnassieux/Scripts/blob/master/PyrapTutorial.ipynb}{an ipython notebook} tutorial on pyrap, a python library. Easily converted into scripts, it has been a very popular tutorial with colleagues over the years.\\
	\multicolumn{2}{c}{} \\

	\textsc{Sep 2017} & \textbf{Rewrote ``Visibility Space" chapter of \emph{Fundamentals of Interferometry}}\\
	& This is an online coursebook written in multiple ipython notebooks, fruit of years of labour from many contributors. Rewrote Julien Girard's work. \hyperlink{https://github.com/ratt-ru/fundamentals_of_interferometry}{Link here}.\\                  
	\multicolumn{2}{c}{} \\

	& \textbf{Lectured Physics 101}\vspace{1mm}\\
	\textsc{Jan 2017} & Introductory undergraduate course in basic mechanics, aimed non-physicist\\
	\textsc{Apr 2017} & undergraduates. Of 60-odd students, 15 were also in my tutorial group.\\
	\multicolumn{2}{c}{} \\	

	
	& \textbf{Paris Observatory ``Parrainage"}\vspace{1mm}\\
	\textsc{Sep 2016} & Paris Observatory’s outreach program, organised by Alain Doressoundiram. I\\
	\textsc{Jul 2015} & helped teachers (primary-school, middle-school) organise scientific demonstrations\\
	\multicolumn{2}{c}{} \\	
	
\end{tabular}

%
%\section{Mentoring \& Supervision}
%
%\begin{tabular}{r|p{12.5cm}}
%	\textsc{2018}    & Helped a PhD student at the University of Bologna, Nadia Biava, with reducing\\
%	\textsc{2020} & data using the LOFAR-Surveys pipeline and with the basics of interferometry, as
%		well as some basics of working on Linux. This involved about an hour of work
%		a week over a period of a few months, as well as 1 paper currently submitted to
%		Astronomy \& Astrophysics (under review).\\
%	\multicolumn{2}{c}{} \\
%	
%	\textsc{2018}    & Helped a PhD student at INAF, Nicola Locatelli, with reducing data using the\\
%	\textsc{2020} & LOFAR-Surveys pipeline and with some basics of interferometry. This involved
%	about an hour of work a week over a period of a few months, and the publication
%	of 1 paper in A\&A.\\
%	\multicolumn{2}{c}{} \\
%	
%	\textsc{2019}    & Helped an MSc student at the University of Bologna, Noemi La Bella, with some
%	basics of working with bash on Linux as well as reducing LOFAR data. This did
%	not result in a publication, though one is in preparation.\\
%	\multicolumn{2}{c}{} \\
%		
%\end{tabular}


%----------------------------------------------------------------------------------------

\end{document}
