%%%%%%%%%%%%%%%%%%%%%%%%%%%%%%%%%%%%%%%%%
% Plasmati Graduate CV
% LaTeX Template
% Version 1.0 (24/3/13)
%
% This template has been downloaded from:
% http://www.LaTeXTemplates.com
%
% Original author:
% Alessandro Plasmati (alessandro.plasmati@gmail.com)
%
% License:
% CC BY-NC-SA 3.0 (http://creativecommons.org/licenses/by-nc-sa/3.0/)
%
% Important note:
% This template needs to be compiled with XeLaTeX.
% The main document font is called Fontin and can be downloaded for free
% from here: http://www.exljbris.com/fontin.html
%
%%%%%%%%%%%%%%%%%%%%%%%%%%%%%%%%%%%%%%%%%

%----------------------------------------------------------------------------------------
%	PACKAGES AND OTHER DOCUMENT CONFIGURATIONS
%----------------------------------------------------------------------------------------

\documentclass[10pt]{article} % Default font size and paper size
\usepackage{geometry}
\geometry{top=2cm,bottom=2cm}
\usepackage{hyperref}


\usepackage{fontspec} % For loading fonts
\defaultfontfeatures{Mapping=tex-text}
%\setmainfont[SmallCapsFont = Fontin SmallCaps]{Fontin} % Main document font


\usepackage{xunicode,xltxtra,url,parskip} % Formatting packages

\usepackage[usenames,dvipsnames]{xcolor} % Required for specifying custom colors

%\usepackage{fullpage}
%\usepackage[big]{layaureo} % Margin formatting of the A4 page, an alternative to layaureo can be \usepackage{fullpage}
% To reduce the height of the top margin uncomment: \addtolength{\voffset}{-1.3cm}
\addtolength{\voffset}{-0.8cm}

\usepackage{hyperref} % Required for adding links	and customizing them
\definecolor{linkcolour}{rgb}{0,0.2,0.6} % Link color
\hypersetup{colorlinks,breaklinks,urlcolor=linkcolour,linkcolor=linkcolour} % Set link colors throughout the document

\usepackage{titlesec} % Used to customize the \section command
\titleformat{\section}{\Large\scshape\raggedright}{}{0em}{}[\titlerule] % Text formatting of sections
\titlespacing{\section}{0pt}{3pt}{3pt} % Spacing around sections

\begin{document}

\pagestyle{empty} % Removes page numbering

%\font\fb=''[cmr10]'' % Change the font of the \LaTeX command under the skills section

%----------------------------------------------------------------------------------------
%	NAME AND CONTACT INFORMATION
%----------------------------------------------------------------------------------------


\par{\centering{\Huge Etienne \textsc{Bonnassieux}}\bigskip\par} % Your name
%\vspace{1 cm}
%%\section{Personal Information}
%\hspace{-0.5cm}
%\begin{tabular}{rl}
%\textsc{Born:} & Noisy-le-Sec  | 19/10/1991 \\
%\textsc{Address:} & 15 Rue Lakanal, 75015 Paris \\
%\end{tabular}
%\hspace{2cm}
\begin{center}
\begin{tabular}{rl|rl}
\textsc{Date of birth:} & 19/10/1991 &
\textsc{email:} & \href{mailto:etienne.bonnassieux@obspm.fr}{etienne.bonnassieux@obspm.fr}\\
\textsc{Place of Birth:} &Noisy-le-Sec &
\textsc{Phone:} & 06 60 63 69 13
\end{tabular}
\end{center}

\vspace{0.5cm}



%----------------------------------------------------------------------------------------
%	EDUCATION
%----------------------------------------------------------------------------------------

\section{Education}

\begin{tabular}{r|p{12.5cm}}
\textsc{2015-2018} & \textbf{PhD in Astrophysics at Observatoire de Paris \& Rhodes University}\\
& ``Statistical Analysis of the Radio-Interferometric Measurement Equation,
a derived adaptive weighting scheme, and applications to LOFAR-VLBI
observation of the Extended Groth Strip"\\
& Partnership: LESIA at the Observatoire de Paris (ED127) \& RATT-RU, SKA-SA\\
\multicolumn{2}{c}{} \\

\textsc{2014-2015} & \textbf{M2R Astronomie, Astrophysique et Ingénierie Spatiale} \\
& Equivalent to Msc in Astrophysics \\
& Partnership : Observatoire de Paris, UPMC, Diderot, Orsay, ENS Ulm\\
%& Université Paris Diderot (Paris 7), Université d'Orsay (Paris 11), \\
%& École Normale Supérieure (ENS Ulm) \\
\multicolumn{2}{c}{} \\

\textsc{2013-2014} & \textbf{M1 Astronomie et Astrophysique, "Sciences de l’Univers et Technologies Spatiales"} \\& - \textit{Observatoire de Paris} \\
& First year of French 2-year Masters. Graduated "Mention Bien" (honours). \\
\multicolumn{2}{c}{} \\

\textsc{2009-2013} & \textbf{Bsc (2:2, Hons) in Astrophysics} - \textit{University of Edinburgh} \\
& Bachelors of Science, graduated with Honours.\\
\multicolumn{2}{c}{} \\

\textsc{2008-2009} & \textbf{IB Diploma} - \textit{Bahrain School} \\
%& Highers: Mathematics, French, Physics, Economics. \\
%& Standard: English, Chemistry \\
\multicolumn{2}{c}{} \\

%------------------------------------------------
\end{tabular}


%----------------------------------------------------------------------------------------
%	WORK EXPERIENCE 
%----------------------------------------------------------------------------------------
%\vspace{0.5 cm}
\section{Research Experience}

\begin{tabular}{r|p{11cm}} 
\textsc{Oct. 2018 - present} & Post-doctoral Fellowship on galactic cluster science at low frequencies at the University of Bologna, under the supervision of Annalisa Bonafede.\\
\multicolumn{2}{l}{Radio Interferometry }\\
Calibration &  During the course of PhD, gained expertise in interferometric calibration techniques and algorithms. This includes use of next-generation tools such as killMS along with older software e.g. Casa or NDPPP, but also work on GPR gain fitting with Landman Bester \& Ulrich Charmel.\\
\multicolumn{2}{c}{} \\
Imaging     &  Also gained expertise in imaging techniques and algorithms, including the use of \& DDFacet and wsclean along with older software e.g. Casa.\\
\multicolumn{2}{c}{} \\
%Low-$\nu$ Astronomy & 
%------------------------------------------------
\end{tabular}


%----------------------------------------------------------------------------------------
%	PUBLICATIONS
%----------------------------------------------------------------------------------------
\section{Publications}

\begin{tabular}{r|p{13.5cm}}
\textsc{Nov.} 2017 & published at Astronomy \& Astrophysics: \hyperlink{https://arxiv.org/abs/1711.00421}{On the variance of radio interferometric calibration solutions}\\
                   & \url{https://arxiv.org/abs/1711.00421}
\end{tabular}


%----------------------------------------------------------------------------------------
%	WORK EXPERIENCE 
%----------------------------------------------------------------------------------------
%\vspace{0.5 cm}
\section{Work Experience}

\begin{tabular}{r|p{12.5cm}} 

%\emph{Mai-Aout 2012} & Stage M1 \\
%\textsc{Mar 2012} & \emph{Etude de la propagation de photons autour d'une singularité de Schwarzschild et de Kerr} \\ 
%& \footnotesize{Calculus} \\
%\multicolumn{2}{c}{} \\

%\textsc{Mar 2015} & \textbf{Masters project under Arache Djannati-Atai - APC, Paris Diderot}\\
%
%%Stage M2 sous la direction de Reza Ansari - Laboratoire de \\
%%& l’accélérateur linéaire \\
%\textsc{Jun 2015} & \emph{Search for >30GeV pulsed signals in HESS-II data. Participation in the implementation of a new event reconstruction algorithm in Tcherenkov imagery.} \\
%& \footnotesize{Still ongoing}.\\
%\multicolumn{2}{c}{} \\
%
%%\textsc{Mar 2015} & Week-long observation project at the Nançay Radioastronomy Station. \\
%%& \footnotesize{Two colleagues and myself collected LoFAR data on a population of pulsars, and determined their Dispersion Measure. We also determined an estimate for the LoFAR pulsar detection threshold. We used data from both the local LoFAR station and the NRT.} \\
%%\multicolumn{2}{c}{} \\
%
%\textsc{Jan 2015} & Informatics project - Observatoire de Paris \\
%& \textit{Characterisation of a chaotic dynamic system, under supervision of Jacques le Bourlot.} \\
%& \footnotesize{The aim of this project was to code a numerical integrator (RK4 with adaptative timestep) and quantify its limitations. This integrator was then used to analyse the evolution of a dynamical system, and quantify its properties using appropriate tools (Lyapunov exponents, bifurcation diagrams, Poincaré cross-sections). The code was written from scratch. This was all done over the course of a week.} \\
%\multicolumn{2}{c}{} \\
%
%%\textsc{Sep 2014} & Informatics project - Observatoire de Paris \\
%%\textsc{Dec 2014} & \emph{Modelling of the accretion disk around a black hole, under supervision of Franck Le Petit} \\ 
%%& \footnotesize{Five of my colleagues and myself simulated the evolution of a black hole's accretion disk. The code was written from scratch. I was responsible for the overall architecture of our code. I thus coordinated the implementation of my colleagues' modules in the main program.} \\
%%\multicolumn{2}{c}{} \\
%
%\textsc{May 2014} & \textbf{Masters project under Misha Haywood - GEPI, Observatoire de Paris} \\
%\textsc{Jun 2014}& \emph{Radial Mixing in Galactic Evolution} \\
%& \footnotesize{Statistical analysis of scientific simulations to quantify the extent of stellar churning in MW-like galaxies. Results were obtained: about 1\% of stars underwent churning of more than 2kpc in the simulation analysed.} \\
%\multicolumn{2}{c}{} \\
%
%\textsc{Oct 2012} & \textbf{Bachelors project under David Lee - UKATC, Edinburgh University} \\
%\textsc{Dec 2012} & \emph{Electroluminescent Panels and LED Panels as Flat-Fielding Apparatus in Astronomical Image Calibration} \\ 
%& \footnotesize{Tested alternative flat-fielding apparatus for use by the MOONS space telescope. In the end, results were conclusive: LED backlights provide better flatness than integrating spheres over a portion of their surface($\sigma_relative=.45\%$ for LED backlights, as opposed to $\sigma_relative=.68\%$), at a fraction of the cost (£20 for backlight used, £1000 for the integrating sphere). Further work was needed to see whether this held true in the infrared, where the MOONS (Multi-Object Optical and Near-infrared Spectrograph) telescope would be observing.} \\
%\multicolumn{2}{c}{} \\

\textsc{Sep 2011} & \textbf{Freelance translator at ISTE Ltd.} \\
\textsc{Feb 2012} & {Translation projects from French to English (mostly engineering books,
ranging from plasma physics to machining). Editing done in Microsoft Word.} \\
\multicolumn{2}{c}{} \\
%------------------------------------------------
\end{tabular}




%----------------------------------------------------------------------------------------
%	TEACHING EXPERIENCE
%----------------------------------------------------------------------------------------
\section{Teaching \& Scientific Outreach}
\hspace{-2cm}
\begin{tabular}{r|p{13.5cm}}
 & \textbf{Tutored in the Paris Observatory DU-LU course}\\
\textsc{Sep 2017 - Jul 2018} & Supervised four students as part of an online course, usually teachers or amateur scientists in the workforce.\\
\textsc{Sep 2015 - Jul 2016} & Of my six students, four successfully carried on to other programs in the DU; two dropped during the year for personal reasons.\\
\multicolumn{2}{c}{} \\


                  & \textbf{Lectured for NASSP}\\
\textsc{Sep 2017} & Taught principles of interferometry: two 1-hour lectures on visibilities, UV-plane, PSF, and ZVC theorem. Course was aimed at honours astrophysics students at UCT.\\
\textsc{Sep 2016} & Same as above, but aimed at masters astrophysics students at UCT: content was at a higher level. This also entailed writing and marking a homework question.\\
\multicolumn{2}{c}{} \\

                  & \textbf{Lectured Physics 101}\\
\textsc{Jan 2017 - Apr 2017} & Introductory undergraduate course in basic mechanics, aimed non-physicist undergraduates. Of 60-odd students, 15 were also in my tutorial group.\\
\multicolumn{2}{c}{} \\
\textsc{Sep 2017} & \textbf{Wrote and organised a pyrap tutorial during \hyperlink{http://www.ast.uct.ac.za/3gc4hifidelity/}{3GC4}}\\
                  & Wrote \hyperlink{https://github.com/ebonnassieux/Scripts/blob/master/PyrapTutorial.ipynb}{an ipython notebook} tutorial on pyrap, a python library. Easily converted into scripts, it has been a very popular tutorial with colleagues over the years.\\
\multicolumn{2}{c}{} \\
\textsc{Sep 2017} & \textbf{Rewrote ``Visibility Space" chapter of \emph{Fundamentals of Interferometry}}\\
                  & This is an online coursebook written in multiple ipython notebooks, fruit of years of labour from many contributors. Rewrote Julien Girard's work. \hyperlink{https://github.com/ratt-ru/fundamentals_of_interferometry}{Link here}.\\                  
\multicolumn{2}{c}{} \\

                             & \textbf{Paris Observatory ``Parrainage"}\\
\textsc{Sep 2015 - Jul 2016} & Paris Observatory's outreach program, organised by Alain Doressoundiram. I helped teachers (primary-school, middle-school) organise scientific demonstrations.\\
\multicolumn{2}{c}{} \\
\end{tabular}




%----------------------------------------------------------------------------------------
%	COMPUTER SKILLS 
%----------------------------------------------------------------------------------------
\vspace{0.35 cm}
\section{Computer Skills}

\begin{tabular}{rp{13.5cm}}
\multicolumn{2}{p{14.5cm}}{Experienced with working under both Linux and Windows environments.}\\% Can touchtype on QWERTY and AZERTY keyboards.}\\
Proficient:  & \textsc{Fortran 70/90/95}, \textsc{C}
, \textsc{bash}, \textsc{LaTeX}, \textsc{Python} \\
%Familiar with: & \textsc{IRAF}, \textsc{Java}, \textsc{Mathematica}\\
%\\
\end{tabular}




%----------------------------------------------------------------------------------------
%	SCHOLARSHIPS AND ADDITIONAL INFO
%----------------------------------------------------------------------------------------

%\section{Scholarships and Awards}
%
%\begin{tabular}{rl}
%\textsc{Sept.} 2009 & President's Education Award for Educational Achievement\\
%
%\textsc{Sept.} 2008 & AP Scholar Award \\

%\textsc{June} 2010 & {\textsc{Gmat}\textregistered}\setmainfont[SmallCapsFont=Fontin SmallCaps]{Fontin-Regular}: 730 (\textsc{q:50;v:39}) %96\textsuperscript{th} percentile; \textsc{awa}: 6.0/6.0 (89\textsuperscript{th} percentile)
%\end{tabular}




%----------------------------------------------------------------------------------------
%   LANGUAGES
%----------------------------------------------------------------------------------------
\vspace{0.5 cm}
\section{Languages}

\begin{tabular}{rl}
\textsc{French} : & Mother Tongue \\
\textsc{English} : & Mother Tongue\\
\textsc{Spanish} : & Proficient

\end{tabular}



%----------------------------------------------------------------------------------------
%	INTERESTS AND ACTIVITIES
%----------------------------------------------------------------------------------------
\vspace{0.5 cm}
\section{Interests and Activities}

%Academic: High-Energy (Astro)Physics, Accelerator Physics, Radio-Astronomy, Interferometry

%Other: History, Sci-fi/Fantasy, Tabletop RPGs/Wargaming, Kickboxing

Student representative for the Astronomy and Astrophysics Masters students at the Observatoire de Paris (and joint establishments) in 2014-2015, student representative for Physics and Astrophysics undergraduates in my year at Edinburgh University in 2010-2011, and on the committee of the Edinburgh Science-Fiction and Fantasy student society in 2010-2011 and 2011-2012 (Librarian, then President, respectively), as well as the Edinburgh University Wargames Society (treasurer).


%----------------------------------------------------------------------------------------

\end{document}
