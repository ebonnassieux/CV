\documentclass[11pt,a4paper,notitlepage]{article}
\usepackage[utf8]{inputenc}
\usepackage[T1]{fontenc}
\usepackage[top=2cm,bottom=2cm]{geometry}
\usepackage{amsmath}
\usepackage{mathtools}
\usepackage{amssymb}
\usepackage{graphicx}
\usepackage{caption}
\usepackage[divpsnames]{xcolor}
\usepackage{subcaption}
\usepackage{hyperref}
%\usepackage{aas_macros}
\usepackage{natbib, twoopt}
\usepackage{booktabs}
\usepackage{floatrow}
\newfloatcommand{capbtabbox}{table}[][\FBwidth]
\providecommand{\e}[1]{\ensuremath{\times 10^{#1}}}
\usepackage{listings}
\usepackage{aas_macros}
\usepackage[capitalise,nameinlink]{cleveref}
%\lstset{language=C,
%  basicstyle=\footnotesize,
%  keywordstyle=\color{blue},
%  commentstyle=\color{green},
%  numberstyle=\color{magenta},
%  stringstyle=\color{blue},
%  identifierstyle=\color{black}
%}
\usepackage{color}
\usepackage{float}
\restylefloat{figure}
\usepackage{changepage}
\usepackage[titles]{tocloft}
\usepackage[justification=centering,font=small,labelsep=period,textfont=it,labelfont=bf]{caption}
\usepackage{url}
\newcommand{\Hrule}{\rule{\linewidth}{0.4mm}}
%\usepackage[french]{babel}
%\interfootnotelinepenalty=10000  % So that long footnotes bring their paragraph with them on next page
\hypersetup{colorlinks=true,linkcolor=blue,citecolor=blue,urlcolor=cyan}
\def\pg{\paragraph{}}


\makeatletter
\newcommandtwoopt{\citeads}[3][][]{\href{http://adsabs.harvard.edu/abs/#3}%
	{\def\hyper@linkstart##1##2{}%
		\let\hyper@linkend\@empty\citealp[#1][#2]{#3}}}
\newcommandtwoopt{\citepads}[3][][]{\href{http://adsabs.harvard.edu/abs/#3}%
	{\def\hyper@linkstart##1##2{}%
		\let\hyper@linkend\@empty\citep[#1][#2]{#3}}}
\newcommandtwoopt{\citetads}[3][][]{\href{http://adsabs.harvard.edu/abs/#3}%
	{\def\hyper@linkstart##1##2{}%
		\let\hyper@linkend\@empty\citet[#1][#2]{#3}}}
\newcommandtwoopt{\citeyearads}[3][][]%
{\href{http://adsabs.harvard.edu/abs/#3}
	{\def\hyper@linkstart##1##2{}%
		\let\hyper@linkend\@empty\citeyear[#1][#2]{#3}}}
\makeatother


\author{Etienne Bonnassieux}
\title{Publications \& Communications}
\date{}

\begin{document}
\maketitle

\section{First-Author Refereed Papers}

\begin{tabular}{r|p{12.5cm}}
%	\textsc{Jul 2021} & in prep. (will submit to Astronomy \& Astrophysics, A\&A - Impact Factor 5.636): Resolved Spectral Properties of 3C295 at LOFAR Frequencies\\
%	\multicolumn{2}{c}{} \\
	
	\textsc{Sep 2021} & Spectral analysis of spatially-resolved 3C295 (sub-arcsecond resolution) with the International LOFAR Telescope \citep{2021arXiv210807294B}\\
	\multicolumn{2}{c}{} \\
	
	\textsc{May 2020} & Decoherence in LOFAR-VLBI beamforming \citep{2020AA...637A..51B}\\
	\multicolumn{2}{c}{} \\
	
	\textsc{Jul 2018} & The variance of radio interferometric calibration solutions. Quality-based weighting schemes \citep{2018AA...615A..66B}\\
	\multicolumn{2}{c}{} \\
	
	%------------------------------------------------
\end{tabular}


\section{Other Refereed Papers}



\begin{tabular}{r|p{12.5cm}}
	
	\textsc{Aug 2021} & A\&A: The resolved jet of 3C 273 at 150 MHz (accepted) \\
	\multicolumn{2}{c}{} \\

	\multicolumn{2}{c}{} \\	\textsc{Aug 2021} & A\&A: Sub-arcsecond imaging with the International LOFAR Telescope I. Foundational calibration strategy and pipeline (accepted) \\
	\multicolumn{2}{c}{} \\
	
	\textsc{Jun 2021} & A\&A: Constraining the AGN duty cycle in the cool-core cluster MS 0735.6+7421 with LOFAR data
	 \citep{2021AA...650A.170B} \\
	\multicolumn{2}{c}{} \\


	\textsc{Feb 2021} & A\&A: Physical insights from the spectrum of the radio halo in MACS J0717.5+3745 \citep{2021AA...646A.135R} \\
	\multicolumn{2}{c}{} \\
	
	\textsc{Feb 2021} & A\&A: Understanding the radio relic emission in the galaxy cluster MACS J0717.5+3745: Spectral analysis \citep{2021AA...646A..56R} \\
	\multicolumn{2}{c}{} \\
	
	\textsc{Jan 2021} & ApJ: The Coma Cluster at LOw Frequency ARray Frequencies. I. Insights into Particle Acceleration Mechanisms in the Radio Bridge \citep{2021ApJ...907...32B} \\
	\multicolumn{2}{c}{} \\
	
	\textsc{Nov 2020} & subm. : LOFAR observations of galaxy clusters in HETDEX \citep{2020arXiv201102387V} \\
	\multicolumn{2}{c}{} \\
	
	\textsc{Nov 2020} & A\&A. : A perfect power-law spectrum even at the highest frequencies: The Toothbrush relic\citep{2020AA...642L..13R} \\
	\multicolumn{2}{c}{} \\
%	
%	\textsc{Sep 2020} &VizieR : VizieR Online Data Catalog: The Toothbrush relic 14.25GHz image c\citep{2020yCat..36429013R} \\
%	\multicolumn{2}{c}{} \\
	
	\textsc{Feb 2019} & A\&A. : The LOFAR Two-metre Sky Survey. II. First data release \citep{2019AA...622A...1S} \\
	\multicolumn{2}{c}{} \\
	
	%------------------------------------------------
\end{tabular}


\section{Posters}

\begin{tabular}{r|p{12.5cm}}
	\textsc{Jun 2017} & \href{https://www.radionet-org.eu/radionet/the-broad-impact-of-low-frequency-observing/}{Broad Impact of Low-Frequency Observing}. Poster on quality-based weighting scheme.\\
	\multicolumn{2}{c}{} \\
\end{tabular}

\section{Conferences \& Workshops}

\begin{tabular}{r|p{12.5cm}}
	\textsc{Mar 2021} & \href{https://www.astron.nl/lofarschool2021/}{6th LOFAR data school}. Talk on direction-dependent calibration, organised hands-on workshop.\\
	\multicolumn{2}{c}{} \\
	
	\textsc{Mar 2021} & \href{https://sites.google.com/inaf.it/rgcw-meeting/home-page}{RGCW Meeting}. Gave talk on progress of NenuFAR Cosmic Filaments \& Magnetic Fields Pilot Surveys.\\
	\multicolumn{2}{c}{} \\
	
	\textsc{Apr 2018} & Invited lecturer at \href{https://indico.ced.inaf.it/event/9/}{the first Italian LOFAR School}. Organised a hands-on workshop on direction-dependent calibration, and helped tutor in the workshops of colleagues.\\
	\multicolumn{2}{c}{} \\
	
	\textsc{Sep 2018} & \href{https://www.astron.nl/lofarschool2018/}{5th LOFAR data school}. Talk on direction-dependent calibration, tutoring hands-on tutorial.\\
	\multicolumn{2}{c}{} \\
	
	\textsc{Dec 2017} & \href{http://www.physics.usyd.edu.au/salf_iv/}{SALF IV}. Gave talk on quality-based weighting schemes.\\
	\multicolumn{2}{c}{} \\
	
	\textsc{Oct 2016} & \href{http://www.ast.uct.ac.za/ast/meetings-workshops/3gc4}{3GC4}. Organised hands-on workshop on \href{https://github.com/ebonnassieux/Teaching/blob/master/PyrapTutorial.ipynb}{using pyrap}, a python wrapper for casacore. This allows scientists to interact with interferometric datasets in an efficient manner.\\
	\multicolumn{2}{c}{} \\
	
\end{tabular}





\bibliographystyle{aa}
\bibliography{biblio}


\end{document}