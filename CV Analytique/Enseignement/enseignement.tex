


\chapter{Exp\'erience d'enseignement} 

\section{Enseignements effectu\'es}

\pg
Mon exp\'erience d'enseignement est d\'ecrite succinctement dans \cref{CV}. J'en r\'esume les caract\'eristiques dans la \cref{tab.teaching}; NASSP signifie "National Astrophysics and Space Science Programme", une structure Sud-Africaine. J'ai \'et\'e charg\'e d'une mission d'enseignement durant ma th\`ese, mais celle-ci ne s'est d\'eroul\'ee qu'a moiti\'e en France; je n'ai donc que 96 heures d'ETD \`a mon actif dans le cadre de cette mission. J'ai compl\'et\'e mon enseignement de ma propre initiative durant mes \'etudes en Afrique du Sud.

\begin{table}[h!]
	\begin{tabular}{|c|c|c|c|}
		\hline
		Enseignement & Institution & Niveau & eHTD \\ \hline 
		Introduction \`a la physique & Rhodes University, Afrique du Sud & L1 & 60 \\ 
		Interf\'erom\'etrie & NASSP, Afrique du Sud & M2 & 15 \\
		Lumi\`eres sur l'Univers & Observatoire de Paris & Dipl\^ome d'Universit\'e & 60 \\
		TP Observation & Unit\'e Scientifique de Nan\c{c}ay & M2 & 20 \\
		Parrainage & Observatoire de Paris & \'Ecole et coll\`ege & 15 \\ \hline
	\end{tabular}\label{tab.teaching}
%\caption{balbla }
\end{table}

\section{Approche p\'edagogique}

\pg
J'ai fait mes \'etudes dans le cadre de plusieurs syst\`emes \'educatifs: syst\`eme Fran\c{c}ais et CNED jusqu'au coll\`ege, syst\`eme Am\'ericain au Bac, licence dans le syst\`eme \'ecossais, et Master \`a l'Observatoire de Paris en France. Chacun a mobilis\'e une approche p\'edagogique propre, en mettant l'emphase sur diff\'erents aspects des concepts enseign\'es.
J'ai de surcro\^it enseign\'e dans le secondaire en Afrique du Sud, au niveau de la L1. Cette exp\'erience a compris \`a la fois encadrement de TDs et cours magistraux. 

\pg
Fort de ces exp\'eriences, je consid\`ere qu'un enseignement efficace demande une forte organisation, qui est une condition n\'ecessaire pour rendre les concepts transmis le plus clair possible, naviguer les doutes et confusions des \'etudiants, et leur permettre de synth\'etiser efficacement leurs connaissances nouvelles avec leur compr\'ehension existante. Bien qu'une telle organisation demande beaucoup de temps, de travail et de discipline, elle permet de faciliter \'enorm\'ement la t\^ache d'enseignement pour tous, professeurs et \'etudiants, une fois men\'ee \`a bien.

\pg
Elle n'est cependant pas suffisante: je consid\`ere qu'il est critique de savoir communiquer \`a ses \'etudiants notre propre int\'er\^et pour la physique, aussi bien sa technique que sa science. Il existe de nombreuses barri\`eres \`a l'assimilation efficace de connaissances: celles-ci peuvent relever de l'ordre de contraintes sociales (e.g. projets d'enfants ou responsabilit\'es de garde de d\'ependants), mat\'erielles (e.g. n\'ecessit\'e de gagner sa vie), psychologiques (e.g. \'episodes de d\'epression chronique), etc. Faire du travail d'assimilation de connaissances un fardeau maximisera les chances que son succ\`es se heurtera \`a ces barri\`eres. Je pense que la meilleure fa\c{c}on de permettre aux \'etudiants de s'approprier r\'eellement le contenu de leurs cours est, d'abord, de leur \textit{donner confiance} en leur propres capacit\'es. 

\pg
La seconde partie du travail d'enseignant revient alors a demander aux \'etudiants de confronter les limites de leur compr\'ehension dans le but d'apprendre. C'est l\`a que l'int\'er\^et pour les sujets abord\'es peut \^etre transmis, que ce soit durant les cours, en travaux dirig\'es ou en travaux pratiques. En communiquant aux \'etudiants que les probl\`emes auxquels ils font face sont r\'esolvables, qu'ils sont capables de les confronter, mais que cela demande un travail dont la gratification est le d\'eveloppement de soi et de leurs connaissances, il devient possible de leur communiquer une motivation propre \`a s'investir dans leur formation, en plus de celles qui les auront pouss\'es \`a la rejoindre (si ce n'\'etait pas leur motivation principale). 

\pg
Mon approche p\'edagogique consiste donc \`a prendre tr\`es au s\'erieux la demande d'organisation impos\'ee par des devoirs d'enseignement. J'{\oe}uvre ensuite \`a permettre aux \'etudiants de se sentir en confiance (e.g. oser poser des questions ``b\^etes'', discuter entre eux des sujets enseign\'es, etc), pour leur communiquer les concepts du cours de fa\c{c}on \`a les encourager \`a devenir moteurs de leur travail. Ceci fait, le travail p\'edagogique devient un travail d'encadrement et de communication. Je note au passage qu'il y aura certainement des \'etudiants qui se sentiront d\'ej\`a en confiance, auront la capacit\'e de travail n\'ecessaire pour mener leur formation \`a bien, et feront des \'etudes formidables. Ma priorit\'e est de permettre aux \'etudiants qui ne font pas encore partie de cette cohorte de la rejoindre en cours de route. Ensuite, personne ne pourra apprendre \`a leur place: je ne peux que les encadrer.


\section{Encadrement de projets de Master et de th\`ese}

\pg
J'ai eu la chance de pouvoir encadrer de nombreux travaux d'\'etudiants de Master et de th\`ese. Durant mon premier poste post-doctoral \`a l'Universit\'e de Bologne, je me suis fait confier l'encadrement de la r\'eduction de donn\'ees LOFAR de plusieurs \'etudiants en Master, Noemi Labella et Giada Pignatora. Cette derni\`ere a poursuivi sa recherche en th\`ese dans la m\^eme \'equipe. J'ai de plus eu la responsabilit\'e d'encadrer la r\'eduction de donn\'ees de deux th\'esards dans la m\^eme p\'eriode, Nadia Biava (qui a soutenu sa th\`ese en 2021) et Giada Pignatora.

\pg
Dans le cadre de mon second poste post-doctoral \`a l'Universit\'e de W\"urzburg, je me suis fait confier l'encadrement du projet de th\`ese de Hrishikesh Shetgaonkar, dont le contrat commen\c{c}ait dans la m\^eme p\'eriode que le mienne. Ma responsabilit\'e rel\`eve \`a la fois de l'encadrement au jour le jour (mise \`a niveau technique, discussions scientifiques, et aide \`a la prise de confiance) et du transfert de connaissances et d'expertise technique (d\'eveloppement de pipelines de r\'eduction, capacit\'e de diagnostic d'images radio-interf\'erom\'etriques, et expertise dans la r\'eduction artisanale de donn\'ees radio). Il entame maintenant sa troisi\`eme ann\'ee de th\`ese, sur un total de quatre pr\'evues, avec deux publications scientifiques en cours de r\'edaction, et ayant pr\'esent\'e ses travaux dans de nombreuses conf\'erences nationales et internationales. 

\pg
Je co-encadre avec Kshitij Thorat une \'etudiante en M2 \`a l'Universit\'e de Pretoria, Katlego Mogamisi, qui s'int\'eresse \`a l'\'etude des interactions entre galaxies et milieux inter-galactiques. Elle soutiendra son m\'emoire de master dans un an, et souhaite poursuivre ses \'etudes avec une th\`ese. Je me suis enfin fait confier l'encadrement de Mehmet Ucak, \'etudiant en Erasmus \`a l'Universit\'e de W\"urzburg, qui a commenc\'e son m\'emoire de Master dans notre \'equipe dans le printemps 2024. J'ai moi-m\^eme \'elabor\'e son projet, qui portera sur le croisement de catalogues d'amas de galaxies (MGCLS, Planck-SZ2) et du catalogue de blazar BZcat. Si des associations sont trouv\'ees, il d\'eterminera s'il s'agit de faux positifs ou bien de blazars r\'eellement situ\'es au sein d'amas de galaxies. 

\section{Insertion dans la Facult\'e de Physique}

\pg
J'ai contact\'e plusieurs personnels de la Facult\'e de Physique (J\'er\^ome Tignon, Pac\^ome Delva, Gwena\"el Bou\'e) afin d'avoir une id\'ee des besoins d'enseignements, ainsi que des projets d\'ej\`a en place. Si ma candidature est retenue, je me plierai \'evidemment aux besoins de la Facult\'e de Physique, en particulier en termes de gestions de groupes de TD et pour l'encadrement du programme de r\'esolution de probl\`emes. Fort de mon exp\'erience au DU-LU de l'Observatoire, je pense que je serai tout \`a fait capable de participer aussi \`a la formation num\'erique de python sur jupyterhubb propos\'ee en L1. Enfin, je peux participer \`a la r\'edaction de documents centraux n\'ecessaires \`a une organisation et une communication efficace entre tout le personnel d'enseignement (polycopi\'es, biblioth\`eque de simulations, cr\'eation d'interfaces utilisateurs agr\'eables, etc), en mettant l'emphase sur la centralit\'e des fondamentaux (physique, math\'ematiques, informatique) qui permettront aux \'etudiants de la Facult\'e de poursuivre sereinement leurs projets professionnels, dans la recherche ou l'industrie.

\pg
Je suis conscient de la charge de travail que repr\'esente la t\^ache d'enseignement du statut de ma\^itre de conf\'erences - m\^eme avec la d\'echarge de premi\`ere ann\'ee prise en compte.

\begin{tcolorbox}[colback=green!10, colframe=green!50!black, arc=3mm, boxrule=1pt]
	Mon profil d'enseignement est adapt\'e aux besoins de la Facult\'e de Physique de Sorbonne-Universit\'e: j'ai \`a la fois une exp\'erience d'encadrement de TD/TP, de suivi num\'erique \`a distance, et d'\'elaboration de documents p\'edagogiques de type jupyter notebook. L'investissement de Sorbonne-Universit\'e dans des m\'ethodes p\'edagogiques novatrices (enseignement agile, atelier de recherche encadr\'ee, r\'esolution de probl\`emes) m'int\'eresse \'enorm\'ement: j'esp\`ere avoir l'occasion de les d\'eployer moi-m\^eme.
\end{tcolorbox}







