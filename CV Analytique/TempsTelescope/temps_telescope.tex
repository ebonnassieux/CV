


\chapter{Temps d'observation obtenus sur t\'elescopes} 

\pg
Mon profil de recherche est celui d'un observateur. J'ai d\'evelopp\'e une capacit\'e \`a identifier des besoins observationnels pour divers projets scientifiques, et d'obtenir le temps d'observation n\'ecessaire. J'ai non seulement obtenu du temps d'observation pour mes propres projets scientifiques, mais aussi en soutien \`a des projets d'\'etudiants en th\`ese.

\pg
J'ai obtenu le temps d'observation en PI (Primary Investigator) sur les projets suivant:
\begin{itemize}
	\item Un total de 60h d'observations NenuFAR dans le cadre du projet LT09 ``Magn\'etisme et Filaments Cosmiques", dont je suis le Principal Investigator.
	\item Un total de 10h d'observations NenuFAR sur M31 (code RP2A), la galaxie d'Androm\`ede, afin de d\'etecter le rayonnement radio-synchrotron provenant de son halo.
	\item Un total de 32h d'observations de M31 avec LOFAR dans sa bande LBA, \`a 60\,MHz, afin de compl\'eter la couverture spectrale de cette source.
	\item Un total de 4h d'observations sur une source myst\'erieuse afin d'en d\'eterminer la nature. Celle-ci se trouvait en bord de champ d'une observation NenuFAR de M31.
	\item Un total de 30h d'observations de M31 avec l'uGMRT (proposal ID 46\_082) .
	\item Un total de 192h sur NenuFAR et sur l'International LOFAR Telescope (ILT, i.e. LOFAR avec lignes de bases internationales) afin de mener \`a bien le commissioning du mode NenuFAR-LSS.
\end{itemize}

\pg
J'ai de plus \'et\'e Co-I sur les projets suivants:
\begin{itemize}
	\item 1Ms d'observations avec INTEGRAL (ID: 2120016) sur M31. Ce projet a \'et\'e men\'e par un coll\`egue, Thomas Siegert, dans le but de compl\'eter en gamma la couverture radio obtenue la m\^eme ann\'ee.
	\item 7h d'observations MeerKAT portant sur des observations de suivi de d\'etections IceCUBE (neutrinos), dans le but d'en d\'eterminer l'origine. Ce projet a \'et\'e \'elabor\'e par moi-m\^eme ainsi qu'un \'etudiant de th\`ese \`a l'Universit\'e de W\"urzburg, Florian R\"osch.
	\item 20h d'observations MeerKAT portant sur le suivi d'un sursaut gamma, dans le but d'en d\'eterminer l'origine par imagerie radio. Ce projet a \'et\'e \'elabor\'e et men\'e par Florian Eppel, un th\'esard \`a l'Universit\'e de W\"urzburg: j'ai servi de soutien et d'expert technique.
	\item 11h d'observations uGMRT sur le blazar 4C+19.44 (proposal ID 46\_081). Ce projet a \'et\'e men\'e par mon \'etudiant en th\`ese, qui \'etudie les n{\oe}uds du jet de ce blazar \`a basses fr\'equences.
	\item 60h d'observations LOFAR-VLBI de blazars \'emettant en TeV (ProjID: LC20\_033), pour un projet men\'e par mon \'etudiant en th\`ese, Hrishikesh Shetgaonkar. Ces observations lui permettront d'\'elaborer une \'etude de population de la contrepartie radio de ces jets gamma.
	\item 26h d'observations LOFAR-HBA d'une population d'amas de galaxies \`a mini-halos (ProjID: LC16\_003), pour un projet men\'e par mon coll\`egue Christopher Riseley afin de d\'eterminer leurs propri\'et\'es spectrales avec LOFAR et MeerKAT. 
	\item 38 d'observations LOFAR-HBA de diff\'erents amas (ProjID: LC15\_021), pour un projet men\'e par ma coll\`egue Kamlesh Rajpurohit dans le but d'\'etudier une classe de reliques radio dont l'\'emissin serait g\'en\'er\'ee principalement par le m\'ecanisme de Diffusive Shock Acceleration.
	\item 80h d'observations LOFAR-HBA de l'amas de Coma (ProjID: LC15\_020), por un projet men\'e par Annalisa Bonafede, cheffe du groupe de recherche dans lequel j'ai travaill\'e pour mon premier post-doc \`a Bologne. J'ai servi d'expertise technique pour ce projet.
	\item 10h d'observations LOFAR-LBA de l'amas cool-core RXJ1720.1+2638 (ProjID: LC12\_018). Ce projet a \'et\'e men\'e par Nadia Biava durant sa th\`ese \`a l'Universit\'e de Bologne.
\end{itemize}

\pg
Je suis donc capable d'obtenir, de fa\c{c}on fiable, du temps d'observation sur des t\'elescopes de classe pr\'ecurseurs SKA, pour mes propres projet, en soutien \`a ceux de mes coll\`egues, ou bien pour des projets de th\`eses.






