
%\font\fb=''[cmr10]'' % Change the font of the \LaTeX command under the skills section

%----------------------------------------------------------------------------------------
%	NAME AND CONTACT INFORMATION
%----------------------------------------------------------------------------------------

\vspace{-4cm}

\chapter{Curriculum Vitae} \label{CV} 

%\par{\centering{\Huge Curriculum Vitae - Etienne \textsc{Bonnassieux}\label{CV}}\bigskip\par} % Your name
\vspace{-0.2cm}
%%\section{Personal Information}
%\hspace{-0.5cm}
%\begin{tabular}{rl}
%\textsc{Born:} & Noisy-le-Sec  | 19/10/1991 \\
%\textsc{Address:} & 15 Rue Lakanal, 75015 Paris \\
%\end{tabular}
%\hspace{2cm}
\begin{center}
\begin{tabular}{rl|rl}
\textsc{Date de naissance:} & 19/10/1991 &
\textsc{email:} & \href{mailto:etienne.bonnassieux@obspm.fr}{etienne.bonnassieux@uni-wuerzburg.de}\\
\textsc{Lieu de naissance:} &Noisy-le-Sec &
\textsc{Phone:} & +33 6 95 98 33 20
\end{tabular}
\end{center}

\vspace{0.3cm}

%----------------------------------------------------------------------------------------
%	COLLABS
%----------------------------------------------------------------------------------------

Mes thématiques de recherche concernent l’étude \`a basses radiofréquences ($\sim30-144$\,MHz) et haute résolution angulaire (sub-arcsec), de la formation et l’évolution des galaxies, de leurs noyaux actifs, et de grandes structures comme les amas de galaxies et les filaments cosmiques avec  l'instrument LOFAR (\textit{LOw Frequency ARray}), et son extension française \textit{NenuFAR} et plus tard \textit{SKA}. J’ai développé une compétence technique et observationnelle en VLBI (\textit{Very Long Baseline Interferometry}) et en radiointerférométrie avancée, adaptée à l’ère \textit{SKA}, permettant la mise en {\oe}uvre de nouvelles techniques et modes d’utilisation de réseaux interféromètriques.




\section{Positions de recherche}

\begin{tabular}{r|p{15.5cm}}
	\textsc{F\'ev 2022} & Poste post-doctoral, portant sur l'\'etude des jets relativistes de blazars avec LOFAR, \`a la \textit{Julius-}\\
	\textsc{Pr\'esent}&\textit{ Maximilians-Universit{\"a}t} de W{\"u}rzburg, Allemagne, sous la supervision de Matthias Kadler dans le cadre du financement \hyperlink{https://www.for5195.uni-wuerzburg.de/}{DFG-FOR5195} en cotutelle avec l'Universit\'e de Hambourg.\\
	\multicolumn{2}{c}{} \\
	\textsc{Oct 2018} & Poste post-doctoral \`a l'Universit\'e de Bologne portant sur l'\'etude des amas de galaxies \`a basses \\
	\textsc{F\'ev 2022}& fr\'equences avec LOFAR, sous la direction d'Annalisa Bonafede dans le cadre de \hyperlink{https://cordis.europa.eu/project/id/714245}{l'ERC DRANOEL}.\\
	\multicolumn{2}{c}{} \\
\end{tabular}


\section{Collaborations internationales pr\'esentes}

\begin{tabular}{r|p{15.5cm}}
	
	\textsc{F\'ev 2022} & \textbf{Unit\'e de recherche DFG: ``Jets Relativistes dans les Galaxies Actives" (Allemagne)}\\
	\textsc{Pr\'esent}  & Employeur actuel. Je travaille sp\'ecifiquement sur l'\'etude de jets blazars \`a larges \'echelles, et ce que nous r\'ev\`elent les observations radios \`a basses fr\'equences sur leur \'emission hautes-\'energies.\\
	\multicolumn{2}{c}{} \\

	
\textsc{Oct 2015} & \textbf{Key Science Projects (KSPs) de LOFAR: Relev\'es et Magn\'etisme (International)}\\
\textsc{Pr\'esent}  &  Groupes de travail : \textbf{Relev\'es}: cartographie le ciel radio Nord \`a 144\,MHz et 60\,MHz. \textbf{Magn\'etisme}: \'etude de la distribution du champ magn\'etique mesur\'e par LOFAR.\\
\multicolumn{2}{c}{} \\

	\textsc{Oct 2017} & \textbf{NenuFAR (France)}\\
	\textsc{Pr\'esent}  & Extension basse-$\nu$ Fran\c{c}aise de LOFAR; je suis le PI du programme long-terme LT09 "Filaments d'amas \& Magn\'etisme Cosmique".\\
	\multicolumn{2}{c}{} \\
	
	\textsc{Oct 2017} & \textbf{Groupe de travail LOFAR-VLBI (International)}\\
	\textsc{Pr\'esent}  &  But: rendre possible, puis faciliter l'utilisation de l'International LOFAR Telescope (r\'esolution sub-arcseconde \`a 144\,MHz).\\
	\multicolumn{2}{c}{} \\
	
	\textsc{Jan 2024} & \textbf{Groupe de travail SKA-VLBI (International)}\\
	\textsc{Pr\'esent}  &  But: d\'evelopper de futures capacit\'es VLBI pour le SKA.\\
	\multicolumn{2}{c}{} \\
	

	
	%------------------------------------------------
\end{tabular}

%----------------------------------------------------------------------------------------
%	EDUCATION
%----------------------------------------------------------------------------------------

\section{Formation \& Qualifications}

\begin{tabular}{r|p{15.5cm}}
	\textsc{Jan 2020} & \textbf{Qualification CNU}\\
	& Section 34.\\
	\multicolumn{2}{c}{} \\

\textsc{2015-2018} & \textbf{Doctorat en Astrophysique} - \textit{Observatoire de Paris $\&$ Rhodes University, Afrique du Sud}\\
& Supervisors: Philippe Zarka, Oleg Smirnov, Cyril Tasse. \textbf{Obtenue en Septembre 2018.}\\
& ``Analyse statistique de l’\'Equation de la Mesure Radio-Interférométrique, un schéma de pondération en découlant, et des applications à une observation LOFAR-VLBI de l’\textit{Extended Groth Strip}"\\
& Co-tutelle: LESIA, Observatoire de Paris (ED127) \& RATT-RU, SKA-SA\\
\multicolumn{2}{c}{} \\

\textsc{2013-2015} & \textbf{M1 \& M2R Astronomie, Astrophysique et Ingénierie Spatiale (Observatoire de Paris)} \\
\multicolumn{2}{c}{} \\


\textsc{2009-2013} & \textbf{Bsc (Hons) in Astrophysics} - \textit{University of Edinburgh} \\


%------------------------------------------------
\end{tabular}


\section{Financements obtenus}

\begin{tabular}{r|p{15.5cm}}
	\textsc{Jul 2023} & \textbf{Financement API-SKA}\\
	& Organisation de la conf\'erence d'\'et\'e du groupe de travail LOFAR-VLBI (21 personnes) \`a l'Observatoire de Paris (750 euros).\\
	\multicolumn{2}{c}{} \\

	%------------------------------------------------
\end{tabular}


\section{Enseignement \& M\'ediation Scientifique}

\begin{tabular}{r|p{15.5cm}}
	\textsc{Jun 2019} & \textbf{Contribution \`a la premi\`ere \'Ecole LOFAR Italienne \`a Bologne}\vspace{1mm}\\
	& Organisation d'un travail pratique sur l'utilsiation de logiciels pour la calibration et l'imagerie d\'ependantes de la direction \textsc{DDFacet}. Participation \`a l'encadrement de cours de r\'eduction de donn\'ees avec \textsc{Prefactor}.\\
	\multicolumn{2}{c}{} \\
	
	& \textbf{Tuteur au DU-LU de l'OBSPM}	\vspace{1mm}\\
	\textsc{Jul 2018} & Suivi de quatre \'etudiants durant la deuxi\`eme moiti\'e de ma th\`ese en France.\vspace{1mm}\\
	\textsc{Sep 2015} & Six \'etudiants durant la premi\`ere moiti\'e.\\
	\multicolumn{2}{c}{} \\

	& \textbf{Enseignement NASSP (University of Western Cape, 15hTD)}\vspace{1mm}\\
	\textsc{Sep 2017} & Cours d'interf\'erom\'etrie: deux cours magistraux d'une heure portant sur l'espace de Fourier, les fonctions de transfert, et le th\'eor\`eme Zernike - van Cittert. Cours destin\'e \`a des \'etudiants en L3.\vspace{1mm}\\
	\textsc{Sep 2016} & Cours d'interf\'erom\'etrie destin\'e aux M2. 10hTD du suivi en continu.\\
	\multicolumn{2}{c}{} \\
	

	\textsc{Sep 2017} & \textbf{Tutoriel de lecture de donn\'ees \`a  \hyperlink{http://www.ast.uct.ac.za/3gc4hifidelity/}{3GC4}}\\
	& R\'edaction d'\hyperlink{https://github.com/ebonnassieux/Scripts/blob/master/PyrapTutorial.ipynb}{un document interactif} montrant l'utilisation d'une librairie python pour la visualisation et la manipulation de donn\'ees interf\'erom\'etriques.\\
	\multicolumn{2}{c}{} \\

	\textsc{Sep 2017} & \textbf{\'Edition du chapitre ``Espace des visibilit\'es" de \emph{Fundamentals of Interferometry}}\\
	& Cours en ligne du RATT-RU (groupe d'interf\'erom\'etrie de \textit{Rhodes University}), \'ecrit sur plusieurs notebooks ipython, et fruit du travail de nombreux contributeurs; %J'ai \'edit\'e le texte de Julien Girard, 
	\hyperlink{https://github.com/ratt-ru/fundamentals_of_interferometry}{lien ici}.\\                  
	\multicolumn{2}{c}{} \\

	& \textbf{Physics 101 (60 hTD, Rhodes University)}\vspace{1mm}\\
	\textsc{Jan 2017} & Cours d'introduction de L1 \`a la m\'ecanique, pour non-physiciens.\\
	\textsc{Apr 2017} & 30hTD Enseignement magistral pour $\sim$60 \'etudiants, et 30hTD de suivi d'une quinzaine d'\'etudiants.\\
	\multicolumn{2}{c}{} \\	

	
	& \textbf{Parrainages de l'Observatoire de Paris (3 classes, 15hTD)}\vspace{1mm}\\
	\textsc{Sep 2016} & Programme de m\'ediation scientifique de l'OBSPM, sous la direction d'Alain Doressoundiram.\\
	\textsc{Jul 2015} & J'ai parrain\'e 3 classes allant d'ULIS (primaire) \`a la seconde.\\
	\multicolumn{2}{c}{} \\	
	
\end{tabular}




\section{Expertise}


\begin{itemize}
	\item Expertise scientifique extra-galactique: jets galactiques, \'evolution de galaxies, milieu inter-galactique, grandes structures.
	\item Sp\'ecificit\'e basses fr\'equences radio: \'etude de rayonnement "fossile".
	\item Expertise en r\'eduction de donn\'ees SKA et pr\'ecurseurs (Big Data); application de techniques VLBI aux instruments pr\'e-SKA.
	\item D\'eveloppement logiciel: impl\'ementation de r\'eponse d'antenne \textit{ATCA} et \textit{NenuFAR} dans des logiciels publics; \textit{containerisation} et d\'eploiement de ces derniers sur de nouvelles architectures de calcul.
	\item Organisation et commissioning instrumental: d\'eveloppement d'imagerie pour l'instrument \textit{NenuFAR}, organisation du plan de \textit{commissioning} de l'utilisation de NenuFAR en tant que super-station LOFAR.
\end{itemize}


%
%\section{Mentoring \& Supervision}
%
%\begin{tabular}{r|p{12.5cm}}
%	\textsc{2018}    & Helped a PhD student at the University of Bologna, Nadia Biava, with reducing\\
%	\textsc{2020} & data using the LOFAR-Surveys pipeline and with the basics of interferometry, as
%		well as some basics of working on Linux. This involved about an hour of work
%		a week over a period of a few months, as well as 1 paper currently submitted to
%		Astronomy \& Astrophysics (under review).\\
%	\multicolumn{2}{c}{} \\
%	
%	\textsc{2018}    & Helped a PhD student at INAF, Nicola Locatelli, with reducing data using the\\
%	\textsc{2020} & LOFAR-Surveys pipeline and with some basics of interferometry. This involved
%	about an hour of work a week over a period of a few months, and the publication
%	of 1 paper in A\&A.\\
%	\multicolumn{2}{c}{} \\
%	
%	\textsc{2019}    & Helped an MSc student at the University of Bologna, Noemi La Bella, with some
%	basics of working with bash on Linux as well as reducing LOFAR data. This did
%	not result in a publication, though one is in preparation.\\
%	\multicolumn{2}{c}{} \\
%		
%\end{tabular}


%----------------------------------------------------------------------------------------
