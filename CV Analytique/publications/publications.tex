


\chapter{Publications Scientifiques}

\section{Articles premier auteur dans revues \`a comit\'e de lecture}

\begin{tabular}{r|p{15cm}}
	
	\textsc{Feb 2022} & \href{https://ui.adsabs.harvard.edu/abs/2022A%26A...658A..10B/abstract}{Spectral analysis of spatially-resolved 3C295 (sub-arcsecond resolution) with the International LOFAR Telescope}. Etienne Bonnassieux, Frits Sweijen, et al. Feb. 2022, A\&A, 658, A10\\
	\multicolumn{2}{c}{} \\
	
	
	\textsc{Nov 2021} & \href{https://ui.adsabs.harvard.edu/abs/2021Galax...9..105B/abstract}{Pilot Study and Early Results of the Cosmic Filaments and Magnetism Survey with Nenufar: The Coma Cluster Field}. Etienne Bonnassieux, Evangelia Tremou, et al. Nov. 2021, Galaxies, 9, 105\\
	\multicolumn{2}{c}{} \\
	
	\textsc{May 2020} & \href{https://ui.adsabs.harvard.edu/abs/2020A%26A...637A..51B/abstract}{Decoherence in LOFAR-VLBI beamforming}. Etienne Bonnassieux, Alastair Edge, et al. May 2020, A\&A, 637, A51\\
	\multicolumn{2}{c}{} \\
	
	\textsc{Jul 2018} & \href{https://ui.adsabs.harvard.edu/abs/2018A%26A...615A..66B/abstract}{The variance of radio interferometric calibration solutions. Quality-based weighting schemes}. Etienne Bonnassieux, Cyril Tasse, et al. Jul. 2018, A\&A, 615, A66\\
	\multicolumn{2}{c}{} \\
	
	%------------------------------------------------
\end{tabular}


\section{Autres articles de revues \`a comit\'e de lecture}



\begin{tabular}{r|p{15cm}}
	
	\textsc{Oct 2023} & \href{https://ui.adsabs.harvard.edu/abs/2023MNRAS.524.6052R/abstract}{A MeerKAT-meets-LOFAR study of Abell 1413: a moderately disturbed non-cool-core cluster hosting a 500 kpc 'mini'-halo }. C. J. Riseley, N. Biava, et al. Oct. 2023, MNRAS, 524, 6052 \\
	\multicolumn{2}{c}{} \\
	
	\textsc{Jan 2023} & \href{https://ui.adsabs.harvard.edu/abs/2023A%26A...669A...1R/abstract}{Deep low-frequency radio observations of Abell 2256. II. The ultra-steep spectrum radio halo}. K. Rajpurohit, E. Osinga, et al. Jan. 2023, A\&A, 669, A1 \\
	\multicolumn{2}{c}{} \\
	
	\textsc{Sep 2022} & \href{https://ui.adsabs.harvard.edu/abs/2022MNRAS.515.1871R/abstract}{Radio fossils, relics, and haloes in Abell 3266: cluster archaeology with ASKAP-EMU and the ATCA}. C. J. Riseley, E. Bonnassieux, et al. Sep. 2022, MNRAS, 515, 1871\\
	\multicolumn{2}{c}{} \\
	
	\textsc{Sep 2022} & \href{https://ui.adsabs.harvard.edu/abs/2022A%26A...665A..60H/abstract}{Diffuse radio emission from non-Planck galaxy clusters in the LoTSS-DR2 fields}. D. N. Hoang, M. Brüggen, et al. Sep. 2022, A\&A, 665, A60 \\
	\multicolumn{2}{c}{} \\
	
	\textsc{Jul 2022} & \href{https://ui.adsabs.harvard.edu/abs/2022ApJ...933..218B/abstract}{The Coma Cluster at LOFAR Frequencies. II. The Halo, Relic, and a New Accretion Relic}. A. Bonafede, G. Brunetti, et al. Jul. 2022, ApJ, 933, 218 \\
	\multicolumn{2}{c}{} \\
	
	\textsc{Jul 2022} & \href{https://ui.adsabs.harvard.edu/abs/2022A%26A...663A..44K/abstract}{Subarcsecond view on the high-redshift blazar GB 1508+5714 by the International LOFAR Telescope}. A. Kappes, P. R. Burd, et al. Jul. 2022, A\&A, 663, A44 \\
	\multicolumn{2}{c}{} \\
	
	\textsc{May 2022} & \href{https://ui.adsabs.harvard.edu/abs/2022MNRAS.512.4210R/abstract}{A MeerKAT-meets-LOFAR study of MS 1455.0 + 2232: a 590 kiloparsec 'mini'-halo in a sloshing cool-core cluster}. C. J. Riseley, K. Rajpurohit, et al. May 2022, MNRAS, 512, 4210 \\
	\multicolumn{2}{c}{} \\
	
	\textsc{May 2022} & \href{https://ui.adsabs.harvard.edu/abs/2022A%26A...661A..92B/abstract}{The galaxy group NGC 507: Newly detected AGN remnant plasma transported by sloshing}. M. Brienza, L. Lovisari, et al. May 2022, A\&A, 661, A92 \\
	\multicolumn{2}{c}{} \\
	
	\textsc{Apr 2022} & \href{https://ui.adsabs.harvard.edu/abs/2022MNRAS.511.3389V/abstract}{Spectral study of the diffuse synchrotron source in the galaxy cluster Abell 523}. Valentina Vacca, Timothy Shimwell, et al. Apr. 2022, MNRAS, 511, 3389\\
	\multicolumn{2}{c}{} \\
	
	
	\textsc{Mar 2022} & \href{https://ui.adsabs.harvard.edu/abs/2022ApJ...927...80R/abstract}{Deep Low-frequency Radio Observations of A2256. I. The Filamentary Radio Relic}. K. Rajpurohit, R. J. van Weeren, et al. Mar. 2022, ApJ, 927, 80 \\
	\multicolumn{2}{c}{} \\
	
	
	\textsc{Mar 2022} & \href{https://ui.adsabs.harvard.edu/abs/2022A%26A...659A...1S/abstract}{The LOFAR Two-metre Sky Survey. V. Second data release}.T. W. Shimwell, M. J. Hardcastle, et al. Mar. 2022, A\&A, 659, A1 \\
	\multicolumn{2}{c}{} \\
	
	\textsc{Feb 2022} & \href{https://ui.adsabs.harvard.edu/abs/2022A%26A...658A...8H/abstract}{The resolved jet of 3C 273 at 150 MHz. Sub-arcsecond imaging with the LOFAR international baselines}. J. J. Harwood, S. Mooney, et al. Feb. 2022, A\&A, 658, A8 \\
	\multicolumn{2}{c}{} \\
	
	
\end{tabular}


\begin{tabular}{r|p{15cm}}
	
	
	\textsc{Feb 2022} & \href{https://ui.adsabs.harvard.edu/abs/2022A%26A...658A...1M/abstract}{Sub-arcsecond imaging with the International LOFAR Telescope. I. Foundational calibration strategy and pipeline}. L. K. Morabito, N. J. Jackson, et al. Feb. 2022, A\&A, 658, A1\\
	\multicolumn{2}{c}{} \\
	
	\textsc{Jan 2022} & \href{https://ui.adsabs.harvard.edu/abs/2022A%26A...657A...2R/abstract}{Turbulent magnetic fields in the merging galaxy cluster MACS J0717.5+3745}. K. Rajpurohit, M. Hoeft, et al. Jan. 2022, A\&A, 657, A2 \\
	\multicolumn{2}{c}{} \\
	
	
	
	\textsc{Oct 2021} & \href{https://ui.adsabs.harvard.edu/abs/2021A%26A...654A..41R/abstract}{Dissecting nonthermal emission in the complex multiple-merger galaxy cluster Abell 2744: Radio and X-ray analysis}. K. Rajpurohit, F. Vazza, et al. Oct. 2021, A\&A, 654, A41\\
	\multicolumn{2}{c}{} \\
	
	
	\textsc{Jul 2021} & \href{https://ui.adsabs.harvard.edu/abs/2021A%26A...651A.115V/abstract}{LOFAR observations of galaxy clusters in HETDEX. Extraction and self-calibration of individual LOFAR targets}. R. J. van Weeren, T. W. Shimwell, et al. Jul. 2021, A\&A, 651, A115\\
	\multicolumn{2}{c}{} \\
	
	\textsc{Jun 2021} & \href{https://ui.adsabs.harvard.edu/abs/2021A%26A...650A.170B/abstract}{Constraining the AGN duty cycle in the cool-core cluster MS 0735.6+7421 with LOFAR data.}  Nadia Biava, Marisa Brienza, et al. Jun. 2021, A\&A, 650, A170 \\
	\multicolumn{2}{c}{} \\
	
	
	\textsc{Feb 2021} & \href{https://ui.adsabs.harvard.edu/abs/2021A%26A...646A.135R/abstract}{Physical insights from the spectrum of the radio halo in MACS J0717.5+3745}. K. Rajpurohit, G. Brunetti, et al. Feb. 2021, A\&A, 646, A135 \\
	\multicolumn{2}{c}{} \\
	
	\textsc{Feb 2021} & \href{https://ui.adsabs.harvard.edu/abs/2021A%26A...646A..56R/abstract}{Understanding the radio relic emission in the galaxy cluster MACS J0717.5+3745: Spectral analysis}. K. Rajpurohit, D. Wittor, et al. Feb. 2021, A\&A, 646, A56 \\
	\multicolumn{2}{c}{} \\
	
	\textsc{Jan 2021} & \href{https://ui.adsabs.harvard.edu/abs/2021ApJ...907...32B/abstract}{The Coma Cluster at LOw Frequency ARray Frequencies. I. Insights into Particle Acceleration Mechanisms in the Radio Bridge}. A. Bonafede, G. Brunetti, et al. Jan. 2021, ApJ, 907, 32 \\
	\multicolumn{2}{c}{} \\
	
	
	\textsc{Nov 2020} & \href{https://ui.adsabs.harvard.edu/abs/2020A%26A...642L..13R/abstract}{A perfect power-law spectrum even at the highest frequencies: The Toothbrush}. K. Rajpurohit, F. Vazza, et al. Oct. 2020, A\&A, 642, L13 \\
	\multicolumn{2}{c}{} \\
	
	\textsc{Apr 2020} & \href{https://ui.adsabs.harvard.edu/abs/2020A%26A...636A..30R/abstract}{New mysteries and challenges from the Toothbrush relic: wideband observations from 550 MHz to 8 GHz}. K. Rajpurohit, M. Hoeft et al, A\&A, Volume 636, id.A30, 20 pp. \\
	\multicolumn{2}{c}{} \\	
	
	
	%	
	%	\textsc{Sep 2020} &VizieR : VizieR Online Data Catalog: The Toothbrush relic 14.25GHz image c\citep{2020yCat..36429013R} \\
	%	\multicolumn{2}{c}{} \\
	
	\textsc{Feb 2019} & \href{https://ui.adsabs.harvard.edu/abs/2019A%26A...622A...1S/abstract}{The LOFAR Two-metre Sky Survey. II. First data release}. T. W. Shimwell, C. Tasse, et al. Feb. 2019, A\&A, 622, A1\\
	\multicolumn{2}{c}{} \\
	
\end{tabular}


\section{Papiers publi\'es sans comit\'e de lecture}

\begin{tabular}{r|p{15cm}}
	\textsc{Nov 2023} & \href{https://ui.adsabs.harvard.edu/abs/2023arXiv231110056K/abstract}{A Collection of German Science Interests in the Next Generation Very Large Array}. M. Kadler, D. A. Riechers, et al. Nov. 2023, arXiv e-prints, arXiv:2311.10056\\
	\multicolumn{2}{c}{} \\
	
	\textsc{Nov 2023} & \href{https://ui.adsabs.harvard.edu/abs/2022arXiv220111526T/abstract}{A distributed computing infrastructure for LOFAR Italian community}. G. Taffoni, U. Becciani, et al. Jan. 2022. arXiv e-prints, arXiv:2201.11526\\
	\multicolumn{2}{c}{} \\
	
	
\end{tabular}

\section{Posters}

\begin{tabular}{r|p{15cm}}
	\textsc{Jun 2017} & \href{https://www.radionet-org.eu/radionet/the-broad-impact-of-low-frequency-observing/}{Broad Impact of Low-Frequency Observing}. Poster (\href{https://github.com/ebonnassieux/CV/blob/master/CV%20Analytique/posters/Bologna%20Poster.pdf}{lien ici}) d\'ecrivant mon travail de th\`ese.\\
	\multicolumn{2}{c}{} \\
\end{tabular}


\section{Conf\'erences \& S\'eminaires Invit\'es}

\begin{tabular}{r|p{15cm}}
	\textsc{Jun 2018} & Pr\'esentation des r\'esultats du groupe LOFAR-VLBI au s\'eminaire de l'Universit\'e de Chalmers \`a Onsala, Su\`ede.\\
	\multicolumn{2}{c}{} \\
	\textsc{Jun 2024} & Pr\'esentation \`a la SF2A des travaux de recherche du groupe LOFAR-VLBI de JMU W\"urzburg, dont je suis co-responsable. Ces derniers portent sur l'imagerie \`a haute r\'esolution du jet d'OJ287.\\
	\multicolumn{2}{c}{} \\
	\textsc{Mar 2024} & Pr\'esentation de mes travaux sur les filaments cosmiques avec LOFAR et NenuFAR au s\'eminaire du LERMA, Observatoire de Paris. \\
	\multicolumn{2}{c}{} \\
	\textsc{Mai 2024} & Pr\'esentation de l'\'etat de l'art du LOFAR-VLBI au Max Planck Institute for Radio Astronomy \`a Bonn, Allemagne.\\
	\multicolumn{2}{c}{} \\
	%\multicolumn{2}{c}{} \\
	\textsc{Nov 2023} & Pr\'esentation de l'\'etat de l'art de LOFAR au Thüringer Landessternwarte \`a Tautenburg, Allemagne.\\
	\multicolumn{2}{c}{} \\
	\textsc{Nov 2022} & Pr\'esentation de \href{https://ui.adsabs.harvard.edu/abs/2022A%26A...658A..10B/abstract}{mes travaux sur 3C295} au s\'eminaire de la chaire d'astronomie de l'Universit\'e de W\"urzburg, en Allemagne.\\
	\multicolumn{2}{c}{} \\
	\textsc{Sep 2018} & Pr\'esentation de mon travail de th\`ese au s\'eminaire de l'Universit\'e de Bologne, Italie.
	%	\multicolumn{2}{c}{} \\
\end{tabular}


\section{Participations orales \`a conf\'erences \& workshops}

\begin{tabular}{r|p{15cm}}
	
	
	
	\textsc{Nov 2023} & \href{https://events.mpifr-bonn.mpg.de/indico/event/324/overview}{GLOW meeting }\`a Bochum. J'y ai pr\'esent\'e mon travail de suivi, avec LOFAR, de relev\'es aux rayons X effectu\'es par Manami Sasaki et Sara Saeedi autour de M31.\\
	\multicolumn{2}{c}{} \\
	
	\textsc{Aug 2023} & \href{https://indico.ecap.work/event/27/}{FRANCI meeting }\`a Bamberg. J'y ai pr\'esent\'e le projet LOFAR-VLBI dans lequel s'inscrit mon travail post-doctoral de W\"urzburg, qui porte sur le suivi de blazars \`a jets \'emettant en X avec le LOFAR-VLBI.\\
	\multicolumn{2}{c}{} \\
	
	\textsc{Jun 2023} & \href{https://www.glowconsortium.de/index.php/en/lofar-family-meeting-2022}{LOFAR Family Meeting} \`a Cologne. J'y ai pr\'esent\'e l'avantage d'une ``super-station'' NenuFAR au sein du International LOFAR Telescope; elle permettrait d'am\'eliorer nettement la calibration et la mesure de quantit\'es de clot\^ure. J'y ai aussi pr\'esent\'e les avantages de nouvelles extensions de l'ILT.\\
	\multicolumn{2}{c}{} \\
	
	\textsc{Mar 2021} & \href{https://www.astron.nl/lofarschool2021/}{6th LOFAR data school}. J'y ai pr\'esent\'e la calibration d\'ependente de la direction pour LOFAR, et organis\'e un hands-on workshop.\\
	\multicolumn{2}{c}{} \\
	
	\textsc{Mar 2021} & \href{https://sites.google.com/inaf.it/rgcw-meeting/home-page}{RGCW Meeting}. J'y ai pr\'esent\'e mes r\'esultats dans le cadre du projet LT09 "Filaments Cosmiques \& Champs Magn\'etiques" de NenuFAR.\\
	\multicolumn{2}{c}{} \\
	
	\textsc{Apr 2018} & Invited lecturer at \href{https://indico.ced.inaf.it/event/9/}{the first Italian LOFAR School}. J'y ai organis\'e un workshop sur la r\'eduction d\'ependante de la direction avec LOFAR, et particip\'e au tutorat dans les workshops de coll\`egues.\\
	\multicolumn{2}{c}{} \\
	
	\textsc{Sep 2018} & \href{https://www.astron.nl/lofarschool2018/}{5th LOFAR data school}. J'y ai donne une contribution orale sur la calibration d\'ependante de la direction, et organis\'e un tutorial hands-on.\\
	\multicolumn{2}{c}{} \\
	
	\textsc{Dec 2017} & \href{http://www.physics.usyd.edu.au/salf_iv/}{SALF IV}. J'y ai pr\'esent\'e mes r\'esultats de th\`ese, un sch\'ema adaptatif de pond\'eration de donn\'ees interf\'erom\`etriques.\\
	\multicolumn{2}{c}{} \\
	
	\textsc{Oct 2016} & \href{http://www.ast.uct.ac.za/ast/meetings-workshops/3gc4}{3GC4}.J'y ai organis\'e un tutoriel portant sur une librairie python,  \href{https://github.com/ebonnassieux/Teaching/blob/master/PyrapTutorial.ipynb}{pyrap}. Celle-ci permet de manipuler facilement des donn\'ees interf\'erom\'etriques.\\
	\multicolumn{2}{c}{} \\
	
\end{tabular}


%
%
%\setlength{\bibsep}{0pt plus 0.3ex}
%%\begin{multicols}{2}
%	\small%jesus 60 MB 
%	{\setstretch{0.5}
	%		\bibliographystyle{bibgen}
	%		\bibliography{biblio}
	%	}
%%\end{multicols}

%
%\bibliographystyle{aa}
%\bibliography{biblio}

