


\chapter{Organisation, administration et valorisation de la recherche} 

\section{Organisation de la vie de recherche}

\pg
Ma participation \`a l'organisation de la vie de recherche se divisie en trois p\'eriodes: ma th\`ese, mon premier post-doc, et mon second post-doc. Durant ma th\`ese, je n'ai commenc\'e \`a \^etre actif sur cet axe que pendant mon s\'ejour en Afrique du Sud, o\`u j'ai \'et\'e l'un des \'etudiants moteurs dans des s\'eances de pr\'esentation de nos travaux et de discussions scientifiques et techniques entre \'etudiants, allant des master aux th\'esards. 

\pg
Mon post-doc \`a Bologne a subi de plein fouet la crise du Covid-19, qui a frapp\'e particuli\`erement durement l'Italie durant la p\'eriode o\`u j'y ai travaill\'e. Notre groupe de recherche regroupant trois post-docs, dont j'\'etais le plus jeune, j'ai particip\'e \`a nos efforts collectifs d'organiser un semblant de vie scientifique en distanciel, et ensuite nos efforts de reprise de vie normale apr\`es le confinement. Ce dernier a n\'eanmoins \'et\'e suffisamment dur pour enrayer assez efficacement ma capacit\'e \`a m'investir efficacement dans cet axe des responsabilit\'es d'un jeune chercheur.

\pg
Mon second post-doc, \`a l'Universit\'e de W\"urzburg en Allemagne, est le premier poste de post-doctorant d\'ecern\'e en radio en Allemagne. On m'a donc confi\'e des responsabilit\'es au-del\'`a de celles d'un simple chercheur et relevant plus des activit\'es d'un professeur junior, y compris l'encadrement direct d'un th\'esard. Pour mener ces nouvelles activit\'es \`a bien, j'ai pris l'initiative d'organiser pendant un an un journal club au sein du groupe de recherche, en y invitant aussi des \'etudiants et chercheurs de toute la facult\'e d'astronomie. J'ai d\^u ensuite passer la main par manque de temps. J'ai cependant continu\'e \`a participer \`a de nombreuses discussions avec les \'etudiants de la facult\'e, afin de les aider \`a s'investir dans leurs projets de th\`ese.

\pg
Enfin, j'ai fait partie du Local Organising Committee (LOC) d'une semaine de travail LOFAR-VLBI \`a l'Observatoire de Paris en \'et\'e 2023. Ce dernier a \'et\'e un succ\`es retentissant aupr\`es des coll\`egues internationaux.


\section{Responsabilit\'es Administratives}

\pg
Dans le cadre de mon post-doc \`a W\"urzburg, l'administratif (tout uniquement en Allemand et en papier) fut l'un des points compliqu\'es \`a g\'erer pour toute la facult\'e. Je me suis coordonn\'e avec le secr\'etaire \`a mi-temps charg\'e de g\'erer l'administratif pour l'ensemble de la facult\'e (2 permanents, 3 profs junior, 4 postdocs, 11) afin de minimiser la lourdeur de la charge. Cela a mobilis\'e un grand temps de travail, de patience et d'\'ecoute, mais \'etait strictement n\'ecessaire pour pouvoir atteindre un fonctionnement minimum pour la facult\'e, car l'administratif de l'ensemble de l'Universit\'e s'est renouvell\'e durant cette p\'eriode. 

\pg
Enfin, en tant que coordinateur de la mise en {\oe}uvre (commissioning) du mode NenuFAR-LSS, j'ai organis\'e une semaine de travail \`a l'Observatoire de Paris. 



\section{Valorisation de la recherche}

\pg
Ayant effectu\'e mes travaux post-doctoraux dans des pays dont je ne parlais initialement pas la langue, je n'ai pas pu m'investir autant que je l'aurai souhait\'e dans des travaux de valorsation de la recherche. J'ai tent\'e de lancer un club d'astronomie amateur au tout d\'ebut de mon postdoc en Allemagne, mais les th\'esards qui \'etaient motiv\'es pour s'y investir ont vite soutenu leur th\`ese, et le projet n'a pas pu aboutir. C'est un aspect du travail de chercheur qui me tient \`a c{\oe}ur et que j'aimerais mener, en coop\'eration avec les structures associatives et la soci\'et\'e civile.

















